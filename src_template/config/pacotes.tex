% Pacotes Fundamentais, favor não alterar nada

% Configuração das margens
\usepackage[inner = 3cm, outer = 2cm, top = 3cm, bottom= 2cm]{geometry}

\usepackage[utf8]{inputenc}
\usepackage[T1]{fontenc}
\usepackage[brazil]{babel}
\usepackage{hyphenat}

% Configuração dos parágrafos
\usepackage{indentfirst}
\setlength{\parindent}{1.25cm}

% Configuração do espaçamento
\usepackage{setspace}

% Configuração de alinhamento de texto
\usepackage{ragged2e}

% Configuração de numeração de página
\usepackage{fancyhdr}
\fancyhf{}
\rhead{\thepage}
\cfoot{}
\renewcommand{\headrulewidth}{0pt} % zera linha de cabeçalho
\pagestyle{fancy}
\setlength{\headheight}{15pt}

% Configuração da Fonte
\usepackage{times} % Fonte: Times New Roman 
%\usepackage{fontspec} % Pacote de Fontes
%\setmainfont{Arial} % Arial - use o compilador XeLaTeX ou LuaLaTeX

\usepackage[hang,flushmargin]{footmisc} % tirar o espaçamento da nota de rodapé

\usepackage{xcolor}

\makeatletter % para primeiras notas de rodapé
\def\@xfootnote[#1]{%
  \protected@xdef\@thefnmark{#1}%
  \@footnotemark\@footnotetext}
\makeatother

% Formatando Sections
\usepackage{mfirstuc}

\usepackage{titlesec}
\titleformat{\section}{\normalfont\fontsize{12}{12}\bfseries\MakeUppercase}{\thesection}{0.5em}{}

\titleformat{\subsection}{\normalfont\fontsize{12}{12}\bfseries}{\thesubsection}{0.5em}{\capitalisewords}

\titleformat{\subsubsection}{\normalfont\fontsize{12}{12}\bfseries\itshape}{\thesubsubsection}{0.5em}{\capitalisewords}


%PACOTE DE CITAÇÕES
\usepackage[alf]{abntex2cite}

% Redefinindo a formatação para usar negrito em vez de itálico
\renewcommand{\emph}[1]{\textbf{#1}}

% Personalização do título das referências
\renewcommand{\refname}{Referências}

% Redefinindo formatação apenas para primeira letra em maiúscula
\newcommand{\ncite}[2][]{(\citeauthoronline{#2}, \citeyear{#2}\ifx&#1&\else, p. #1\fi)}

\usepackage{changepage}
\usepackage{ifoddpage}

% Definir ambiente quote para citações diretas com mais de 3 linhas
\newenvironment{myquote}{\par\fontsize{10pt}{12pt}\selectfont\begin{adjustwidth}{4cm}{0pt}}{\end{adjustwidth}\par}

% PACOTES ESSENCIAIS
\usepackage{graphicx}
\usepackage{float}
\usepackage{caption}
\usepackage{xurl}

% Define o tamanho da fonte para as legendas
\DeclareCaptionFont{mycaptionfont}{\fontsize{10pt}{12pt}\selectfont}

% Aplica o tamanho da fonte definido às legendas de figuras
\captionsetup{font=mycaptionfont}

% Redefine o separador das legendas para travessão (-)
\DeclareCaptionLabelSeparator{myhyphen}{ --- }
\captionsetup{font=mycaptionfont, labelsep=myhyphen}

% Define o espaçamento horizontal da nota de rodapé
\setlength{\footnotemargin}{10pt}
