% IFRS - Campus Restinga
% Adaptado por Jonathan Jardim Nefoussi

% Este template foi adaptado para o layout exigido pelo IFRS Campus Restinga com base no template original do 
% INSTITUTO FEDERAL DE EDUCAÇÃO, CIÊNCIA E TECNOLOGIA DE BRASÍLIA
%
% Template de Artigo Científico
% Conforme modelo do Manual de Normalização de Trabalhos Acadêmicos
%
% Desenvolvido por Lucas Santos de Oliveira
% E-mail: lucas.oliveira4@estudante.ifb.edu.br
%
% ===============================================================

\documentclass[a4paper, 12pt, twoside]{article}

\input{config/pacotes} % Não apague essa linha

% Preencha as informações do título, subtítulo, autor e orientador aqui
\newcommand{\titulo}{[TÍTULO DO ARTIGO NO IDIOMA DO DOCUMENTO]:}
\newcommand{\subtitulo}{[subtítulo (se houver)]}
\newcommand{\autor}{[Nome completo do(a) autor(a)]}
\newcommand{\orientador}{[Nome completo do(a) orientador(a), (se houver)]}
\newcommand{\coorientador}{[Nome completo do(a) coorientador(a), (se houver)]}
\newcommand{\tituloingles}{[OPCIONAL] [TÍTULO DO ARTIGO EM OUTRO IDIOMA]:}
\newcommand{\subtituloingles}{[subtítulo (se houver) em língua estrangeira]}
\newcommand{\aprovacao}{Data de aprovação: XX/XX/XXXX}

% Inclua os pacotes necessários nas linhas abaixo
\usepackage{amsfonts,amsmath,amssymb}

\begin{document}
\singlespacing % espaçamento simples
\justifying % texto justificado

% Títulos e Subtítulos

% caso não tenha subtítulo, coloque % antes do \par \MakeLowercase{\subtitulo}
% Caso não queira utilizar um título em outro idioma, coloque % antes do \par \textbf{\MakeUppercase{\tituloingles}}
% Caso não tenha um subtítulo em outro idioma, coloque % antes do \par \MakeLowercase{\subtituloingles}

\begin{center}

    \par \textbf{\MakeUppercase{\titulo}} 
    
    \par \MakeLowercase{\subtitulo}
    
\end{center}
 % Insere os títulos e subtítulos

%%%%%%%%%%%%%%%%%%%%% RESUMO E ABSTRACT %%%%%%%%%%%%%%%%%%%%%%%%%%%%%%

\begin{flushright}

    \par \autor\footnote[*]{Breve currículo que qualifique o autor na área de conhecimento do artigo. O currículo, bem como o endereço eletrônico. Ex: Pós-graduanda em Ensino de humanidades no Instituto Federal de Ciência, Tecnologia e Educação de Brasília — Campus Gama. Bacharela em Pedagogia pela Universidade de Brasília. E-mail: fulana.tal@ifb.edu.br.}

    \par \orientador\footnote[**]{} 

    \par \coorientador\footnote[***]{} \medskip
    
\end{flushright}

\begin{center}

    \par \textbf{\MakeUppercase{Resumo}}
    
\end{center}

\noindent [OBRIGATÓRIO] Resumo é a apresentação concisa dos pontos relevantes de um documento. Deve-se ressaltar o objetivo, o método, os resultados e as conclusões do documento. É uma sequência de frases concisas, afirmativas e não de enumeração de tópicos. Deve ser apresentado em um parágrafo, com alinhamento justificado com espaçamento entre linhas simples. Deve conter a explicação do tema principal do documento e a informação sobre a categoria do tratamento (memória, estudo de caso, análise da situação, etc.). O verbo deve estar na voz ativa e na terceira pessoa do singular. A extensão do resumo deve ser de 100 a 250 palavras. As palavras-chave devem ser grafadas com as iniciais em letra minúscula, exceto os substantivos próprios e nomes científicos (ABNT, 2021, p. 2). \medskip

\noindent \textbf{Palavras-chave:} palavra-chave 1; palavra-chave 2; palavra-chave 3. \bigskip


%%%%%%%%%%%%%%%%%%%%%%%%%%%%%% INTRODUÇÃO %%%%%%%%%%%%%%%%%%%%%%%%%%%%%

\section{Introdução}

\par [OBRIGATÓRIO] É parte inicial do artigo. Nela deve constar a delimitação do assunto tratado, os objetivos da pesquisa e outros elementos necessários para situar o tema do artigo.

\par Lembre-se de seguir as regras de citações diretas e indiretas constantes na NBR 10520 (citações). Use o arquivos \textbf{\textit{refbib.bib}} contidos nesse template para que as referências sejam processadas pelo pacote \textit{\textbf{abntex2cite}} .

\par Quaisquer outras orientações de formatação estão disponíveis conforme referência no arquivo \textbf{\textit{\_Orientações.pdf}}

%%%%%%%%%%%%%%%%%%%%%%%%%%%%%% DESENVOLVIMENTO %%%%%%%%%%%%%%%%%%%%%%%

\section{Desenvolvimento}

\par [OBRIGATÓRIO] Parte que contém a exposição ordenada e pormenorizada do assunto tratado. Deve ser dividido em seções e subseções, conforme a NBR 6024 (ABNT, 2012).

%%%%%%%%%%%%%%%%%%%%%%%%%%%% CONSIDERAÇÕES FINAIS %%%%%%%%%%%%%%%%%%%%%%

\section{Considerações Finais}

\par [OBRIGATÓRIO] Parte final do artigo, na qual se apresentam as considerações correspondentes aos objetivos e/ou hipóteses.

%%%%%%%%%%%%%%%%%%%%%%%%%% REFERÊNCIAS %%%%%%%%%%%%%%%%%%%%%%%%%%%%%%%%
% Adicionar as referências no arquivo refbib.bib

% Não altere nenhum comando neste arquivo

\titleformat{\section}[block]{\normalfont\fontsize{12}{12}\bfseries\filcenter}{\thesection}{1em}{\MakeUppercase}

\bibliography{refbib}


%%%%%%%%%%%%%%% 

\section*{Apêndice A - Título do Apêndice} 

\par [OPCIONAL] Texto ou documento elaborado pelo autor, a fim de complementar sua argumentação, sem prejuízo da unidade nuclear do trabalho.

\section*{Anexo A - Título do Anexo}

\par [OPCIONAL] Texto ou documento não elaborado pelo autor, que serve de fundamentação, comprovação e ilustração.

\end{document}